\documentclass{weeklyreport}
\usepackage[utf8]{inputenc}

%%% Template Usage
% 1. Go to "All Projects" and make a copy in Overleaf,
% or download the source to modify locally.
% 2. Fill in your name
% 3. Set the reportdate to Monday of the current week


\title{Progress Report}
\author{Edris Qarghah}

\DTMsavedate{reportdate}{2020-08-03}
\date{Week of \DTMusedate{reportdate}}

\begin{document}

\maketitle

\newpage

\section*{Weekly Goals}


\begin{itemize}
	\item Finalize tweaking of network distributions
	\item Create means of visualizing individual path changes
	\item Observe impacts of breaking an edge:
	\begin{itemize}
		\item Use traceroutes to identify path changes
		\item Identify edges that are no longer being used
		\item Infer identity of broken edge
		\item Display impact on latency/packetloss
	\end{itemize}
	\item Observe impact of breaking multiple edges
\end{itemize}

\section*{Daily Log}

\subsection*{\weekday{0}}

\begin{itemize}
    \item Network Analytics Discussion meeting.
    \item Made temporary histograms to confirm distribution matching.
    \begin{itemize}
		\item Loaded histogram data to dataframes and scaled.
		\item Tried alternative means of calculating latency (using the distribution).
		\item Adjust packet-loss to account for the probable number of multiplications.
		\item Added path hops, latency and packetloss to path lengths function.
		\item Updated OWD distribution to account for offset.
	\end{itemize}
    \item Finished distribution matching (got Ilija's sign-off).
    \item Fixed issue with lists containing one tuple being loaded as just a tuple.
\end{itemize}

\subsection*{\weekday{1}}

\begin{itemize}
    \item Added function to change edge values (i.e., latency, packetloss).
    \item Manually changed latency and noted change in shortest paths.
    \item Added shortest paths dictionary to the graph properties.
    \begin{itemize}
    	\item Encountered issues saving and loading shortest path dictionaries (GML does not like taking non-strings as dictionary keys).
    \end{itemize}
    \item Created means of converting a shortest path list to a list of edges.
    \item Created preliminary means of getting changes to paths.
    \item Added capacity to only draw paths connecting specific PS pairs.
\end{itemize}


\subsection*{\weekday{2}}

\begin{itemize}
    \item Added helper functions to address issue with non-string keys (i.e., pair to string, string to pair).
    \item Improved graph path change function to account for bi-directionality of edges.
    \item Met with Ilija about results of killing individual edges. Suggestions:
    \begin{itemize}
    	\item Kill edges entirely (rather than simply increasing latency)
    	\item Systematically remove each individual edge in a graph (this may break the graph into multiple components).
    \end{itemize}
    \item Prepared slide for OSG update meeting.
    \item Read about \href{https://multicoin.capital/2019/07/16/the-separation-of-time-and-state/}{time clock sync issues in blockchain and distributed systems}.
    \item Ensured that path comparisons still functioned if path could not be completed after edge removal.
\end{itemize}

\subsection*{\weekday{3}}

\begin{itemize}
	\item Reviewed OSG update meeting presentation.
	\item SAND Informal Working Meeting. 
	\begin{itemize}
		\item Learned more about what the others are doing:
		\begin{itemize}
			\item Tommy (UMich undergrad) is working on Kibana dashboards for network monitoring
			\item Manjari (UMich undergrad) is working on a UI for network topology analysis (highly relevant to my work)
			\item Petya (U Plovdiv PhD candidate) is working on a Plotly Dash ML web app to provide site summaries.
		\end{itemize}
		\item Tommy helped Petya with data pulling issues in ElasticSearch, which was instructive (she was pulling a set of IPs first as a source, then as a destination but was mysteriously getting the same result both ways).
	\end{itemize}
	\item Wrote functions to systematically remove each edge in a network then tally the resulting path changes.
	\item Created histograms to display edge ambiguity.
	\item Modified packetloss histogram to allow comparison of multiple iterations of graphs.
	\item Updated hops histogram to actually be a histogram!
	\item Patched issue with histogram obscuring original distribution by setting alpha to .5. A better solution needs to be found.
\end{itemize}


\subsection*{\weekday{4}}

\begin{itemize}
	\item Finished reports.
	\item Caught an exception related to how shortest paths were being stored.
	\item When getting path changes, fixed how edges were being removed from the changed\_edges dictionary.
	\item Updated graph drawing:
	\begin{itemize}
		\item Adjusted perfsonar node color to make them easier to see (blue instead of brown).
		\item Changed the default edge label to None (easier to see paths/colors that way).
	\end{itemize}
	\item Add naming and autosaving of histograms.
	\item Updated slides following meeting to clear up ambiguity and structure them better.
	\item Added details about the process for determining deleted edges:
	\begin{enumerate}
		\item Determine shortest paths between all PerfSONAR (PS) nodes in G.
		\item Delete an edge and save the network as a new graph, H.
		\item Determine shortest paths between PS nodes in H.
		\item Compare shortest paths between graphs:
		\begin{enumerate}
			\item Record edges in each G path that aren’t in the corresponding H path (i.e., edges that are potentially the removed edge).
			\item Confirm those edges were removed entirely from H (i.e., they are not on some other path in H).
			\item Determine how many different paths each edge was removed from.
			\begin{enumerate}
				\item It’s likely that the edge removed from the most paths is the deleted edge.
				\item If multiple edges were removed an equal number of times, there is ambiguity as to which was deleted.
			\end{enumerate}
		\end{enumerate}
		\item Repeat 2-4 for every edge in G.
	\end{enumerate}
	\item Caught and corrected issue with edge ambiguity (wasn't fully excluding edges that appear elsewhere in the output graph). This involved a significant rewrite of get\_path\_changes.
	\item Fixed font size on histograms.
\end{itemize}

\section*{Achievements}
\subsection*{I learned about...}
\begin{itemize}
	\item The \href{https://multicoin.capital/2019/07/16/the-separation-of-time-and-state/}{“clock problem” for blockchain/“timechain” and distributed systems}.
	\item Other students on the SAND team and how their work relates:
	\begin{itemize}
		\item Tommy (UMich undergrad) is working on Kibana dashboards for network monitoring
		\item Manjari (UMich undergrad) is working on a UI for network topology analysis (highly relevant to my work)
		\item Petya (U Plovdiv PhD candidate) is working on a Plotly Dash ML web app to provide site summaries.
	\end{itemize}
\end{itemize}

\subsection*{I created...}
\begin{itemize}
	\item Means of identifying changed paths:
	\begin{itemize}
		\item Edges difference between old/new trace-routes.
		\item Accounting for lack of edge directionality.
	\end{itemize}
	\item Means of narrowing down candidate edges:
	\begin{itemize}
		\item Discarding edges that remain on other paths.
		\item Discarding edges not removed from all paths (only works if we know only one. edge)	
	\end{itemize}
	\item Means of visualizing changes:
	\begin{itemize}
		\item Graphs highlighting only relevant paths.
		\item Histograms of candidate edges and impact to distributions.	
	\end{itemize}
\end{itemize}

\pagebreak
\section*{Roadblocks}

\subsection*{Questions}

\begin{itemize}
	\item How can I narrow down the number of potentially impacted edges?
	\item Is there a way to leverage edge adjacency to help make these determinations?
\end{itemize}

\subsection*{Problems}

\begin{itemize}
	\item Edges are indexed as tuples, making their use as keys in dictionaries difficult (ordering and saving as strings).
	\item The toy network has ambiguity with even very limited curated data, which will only be worse with real data.
\end{itemize}

\subsection*{Challenges}

\begin{itemize}
    \item Highlighting changes to a graph without things getting too cumbersome.
	\item Developing ways to summarize the impact of changes over many iterations that were intuitive/easy to understand.
\end{itemize}

\section*{Plans for Next Week}

\begin{itemize}
	\item Toy Network
	\begin{itemize}
		\item Further static network tomography:
		\begin{itemize}
			\item Dead vs. throttled edges
			\item Adjusting for change to variable number of edges
			\item Add bandwidth and measure bandwidth impact
			\item Edge removal heatmap
		\end{itemize}
		\item Simulate network activity
		\item Measure activity from specific nodes/generate Time Series
		\item Develop tools for tomography and anomaly detection
	\end{itemize}
	\item Real Network
	\begin{itemize}
		\item Take snapshots of traceroutes at various times for comparison.
		\item Try to determine whether traceroute variation can be used in the same manner as in toy network to identify dead/ailing edges.
	\end{itemize}
\end{itemize}

\end{document}
